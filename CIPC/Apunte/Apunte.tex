\documentclass[11pt]{article}
\usepackage{amsmath}  % Math
\usepackage{amssymb}  % Symbols
\usepackage{graphicx} % Images
\usepackage[utf8]{inputenc}
\usepackage[T1]{fontenc}
\usepackage[margin=1in]{geometry}
\usepackage[spanish]{babel} % Spanish language support (ACTIVADO)
% --- Ruta de las imágenes ---
\graphicspath{ {./images/}} 

\title{Apunte de Programación en C++ \\ Nivel Inicial}
\author{Felipe Colli \thanks{CIPC}}
\date{\today}

\begin{document}
\maketitle

% --- Portada con imagen ---
\begin{figure}[htbp]
    \centering
    \includegraphics[width=0.8\textwidth]{gatito}
    \caption{Portada de los apuntes.} % Se agregó una descripción a la imagen
\end{figure}

\newpage
\tableofcontents % Esto generará el índice automáticamente
\newpage


\section{Día 1 (Martes, 22/07/2025)}
antes qde comenzra, que es un algoritmo?
es una secuencia de instrucciones que resuleven un porblema
\subsection{Introduccion a C++}
\subsubsection{Porque suamos C++?}
\begin{itemize}
    \item Lengujae de bajo nivle
    \item Grna velocidad (casi igual a C puro)
    \item Biblioteca estandar suoercompleta (STL)
    \item Es el lenguaje ominate de los jueces online
\end{itemize}

%insretar plantilla minima de c++, que tengo como papa en un snippet

\subsubsection{Tipos Basicos de variables}
\begin{itemize}
    \item Int ($\pm 2 \cdot 10^9$)
    \item Long Long (se define como ll normlamnete )($\pm 9 \cdot 10^18$)
    \item Double (decimales) (15 digitos decimales)
    \item char (unaletra o num, solo uno) (1 byte)
    \item string (cadena de chars (tamaño dinamico))
    \item Vector (Arreglo de tamnaño dinamico)
    \item Array (Arreglo de tamaño fijo)
\end{itemize}



\subsection{Complejidad}
En Prog Comp hya limites de tiempos estrictos, y los programas encason que excedan este limite nos daran de veredicto \textbf{TLE (Time Limit Excedeed)}

Como no podemos detereminar el tiempo excato, detrminamos cuanto crece la funcion
para eso suamos la notacion asintotica \textbf{Big O}

Como La calculamos:
en base al input, determinamos cuntas operaciones relaiza

una vez calculamos la compeljidad, esat se deivide por $10^8$ 
\[Tiempo=\frac{f(n)}{10^8} \]


\newpage


\section{Día 2 (Miércoles, 23/07/2025)}
Contenido del segundo día...

\newpage
\section{Día 3 (Jueves, 24/07/2025)}
Contenido del tercer día...

\newpage
\section{Día 4 (Viernes, 25/07/2025)}
Contenido del cuarto día...

\newpage
\section{Día 5 (Sábado, 26/07/2025)}
Contenido del quinto día...

\newpage
\section{Día 6 (Lunes, 28/07/2025)}
Contenido del sexto día...

\newpage
\section{Día 7 (Martes, 29/07/2025)}
Contenido del séptimo día...

\newpage
\section{Día 8 (Miércoles, 30/07/2025)}
Contenido del octavo día...

\newpage
\section{Día 9 (Jueves, 31/07/2025)}
Contenido del noveno día...

\newpage
\section{Día 10 (Viernes, 01/08/2025)}
Contenido del décimo y último día de apuntes.

\end{document}
